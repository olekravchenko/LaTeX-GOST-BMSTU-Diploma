% !TEX program = xelatex
%----------------------- Преамбула -----------------------
\documentclass[utf8x, 14pt, oneside, a4paper]{article}

\usepackage{extsizes} % Для добавления в параметры класса документа 14pt

% Для работы с несколькими языками и шрифтом Times New Roman по-умолчанию
\usepackage[english,russian]{babel}
\usepackage{fontspec}
\setmainfont{Times New Roman}

% ГОСТовские настройки для полей и абзацев
\usepackage[left=30mm,right=15mm,top=20mm,bottom=20mm]{geometry}
\usepackage{misccorr}
\usepackage{indentfirst}
\usepackage{enumitem}
\setlength{\parindent}{1.25cm}
\linespread{1.3}
\setlist{nolistsep} % Отсутствие отступов между элементами \enumerate и \itemize

% Дополнительное окружения для подписей
\usepackage{array}
\newenvironment{signstabular}[1][1]{
	\renewcommand*{\arraystretch}{#1}
	\tabular
}{
	\endtabular
}

% Переопределение стандартных \section, \subsection, \subsubsection по ГОСТу;
% Переопределение их отступов до и после для 1.5 интервала во всем документе
\usepackage{titlesec}

\titleformat{\section}[block]
{\bfseries\normalsize\filcenter}{\thesection}{1em}{}

\titleformat{\subsection}[hang]
{\bfseries\normalsize}{\thesubsection}{1em}{}
\titlespacing\subsection{\parindent}{\parskip}{\parskip}

\titleformat{\subsubsection}[hang]
{\bfseries\normalsize}{\thesubsubsection}{1em}{}
\titlespacing\subsubsection{\parindent}{\parskip}{\parskip}

% Работа с изображениями и таблицами; переопределение названий по ГОСТу
\usepackage{caption}
\captionsetup[figure]{name={Рисунок},labelsep=endash}
\captionsetup[table]{singlelinecheck=false, labelsep=endash}

\usepackage{graphicx}
\usepackage{slashbox} % Диагональное разделение первой ячейки в таблицах

% Цвета для гиперссылок и листингов
\usepackage{color}

% Гиперссылки \toc с кликабельностью
\usepackage{hyperref}

\hypersetup{
	linktoc=all,
	linkcolor=black,
	colorlinks=true,
}

% Листинги
\setsansfont{Arial}
\setmonofont{Courier New}

\usepackage{color} % Цвета для гиперссылок и листингов
\definecolor{comment}{rgb}{0,0.5,0}
\definecolor{plain}{rgb}{0.2,0.2,0.2}
\definecolor{string}{rgb}{0.91,0.45,0.32}

\usepackage{listings}
\lstset{
	basicstyle=\footnotesize\ttfamily,
	language=[Sharp]C, % Или другой ваш язык -- см. документацию пакета
	commentstyle=\color{comment},
	numbers=left,
	numberstyle=\tiny\color{plain},
	numbersep=5pt,
	tabsize=4,
	extendedchars=\true,
	breaklines=true,
	keywordstyle=\color{blue},
	frame=b,
	stringstyle=\ttfamily\color{string}\ttfamily,
	showspaces=false,
	showtabs=false,
	xleftmargin=17pt,
	framexleftmargin=17pt,
	framexrightmargin=5pt,
	framexbottommargin=4pt,
	showstringspaces=false,
	inputencoding=utf8x,
	keepspaces=true
}

\DeclareCaptionLabelSeparator{line}{\ --\ }
\DeclareCaptionFont{white}{\color{white}}
\DeclareCaptionFormat{listing}{\colorbox[cmyk]{0.43,0.35,0.35,0.01}{\parbox{\textwidth}{\hspace{15pt}#1#2#3}}}
\captionsetup[lstlisting]{
	format=listing,
	labelfont=white,
	textfont=white,
	singlelinecheck=false,
	margin=0pt,
	font={bf,footnotesize},
	labelsep=line
}


% Годные пакеты для обычных действий
\usepackage{ulem} % Нормальное нижнее подчеркивание
\usepackage{hhline} % Двойная горизонтальная линия в таблицах
\usepackage[figure,table]{totalcount} % Подсчет изображений, таблиц
\usepackage{rotating} % Поворот изображения вместе с названием
\usepackage{lastpage} % Для подсчета числа страниц

% ---------------------- Документ ----------------------
\begin{document}
	\begin{titlepage}
		\noindent\begin{minipage}{0.05\textwidth}
			\includegraphics[scale=0.3]{1.png}
		\end{minipage}
		\hfill
		\begin{minipage}{0.85\textwidth}\raggedleft
			\begin{center}
				\fontsize{12pt}{0.3\baselineskip}\selectfont \textbf{Министерство науки и высшего образования Российской Федерации \\ Федеральное государственное бюджетное образовательное учреждение \\ высшего образования \\ <<Московский государственный технический университет \\ имени Н.Э. Баумана \\ (национальный исследовательский университет)>> \\ (МГТУ им. Н.Э. Баумана)}
			\end{center}
		\end{minipage}

		\begin{center}
			\fontsize{12pt}{0.1\baselineskip}\selectfont
			\noindent\makebox[\linewidth]{\rule{\textwidth}{4pt}} \makebox[\linewidth]{\rule{\textwidth}{1pt}}
		\end{center}

		\begin{flushleft}
			\fontsize{12pt}{0.8\baselineskip}\selectfont 
			
			ФАКУЛЬТЕТ \uline{<<\textbf{<Факультет (полностью)>}>> \hfill}

			КАФЕДРА \uline{\hspace{4mm} <<\textbf{<Кафедра (полностью)>}>> \hfill}
		\end{flushleft}

		\vfill

		\begin{center}
			\fontsize{20pt}{\baselineskip}\selectfont

			\textbf{РАСЧЕТНО-ПОЯСНИТЕЛЬНАЯ ЗАПИСКА}

			\textbf{\textit{К ВЫПУСКНОЙ КВАЛИФИКАЦИОННОЙ РАБОТЕ}}

			\textbf{\textit{НА ТЕМУ:}}
		\end{center}

		\begin{center}
			\fontsize{18pt}{0.6cm}\selectfont 
			
			\uline{\hfill}
	
			\uline{\hfill}
	
			\uline{\hfill}
	
			\uline{\hfill}
	
			\uline{\hfill}
		\end{center}

		\vfill

		\begin{table}[h!]
			\fontsize{12pt}{0.7\baselineskip}\selectfont
			\centering
			\begin{signstabular}[0.7]{p{7.25cm} >{\centering\arraybackslash}p{4cm} >{\centering\arraybackslash}p{4cm}}
				Студент группы \textbf{<Группа (Шифр)>} & \uline{\hspace*{4cm}} & \uline{\hfill \textbf{<И.О. Фамилия>} \hfill} \\
				& \scriptsize (Подпись, дата) & \scriptsize (И.О. Фамилия)
			\end{signstabular}

			\vspace{\baselineskip}

			\begin{signstabular}[0.7]{p{7.25cm} >{\centering\arraybackslash}p{4cm} >{\centering\arraybackslash}p{4cm}}
				Руководитель ВКР & \uline{\hspace*{4cm}} & \uline{\hfill \textbf{<И.О. Фамилия>} \hfill} \\
				& \scriptsize (Подпись, дата) & \scriptsize (И.О. Фамилия)
			\end{signstabular}

			\vspace{\baselineskip}

			\begin{signstabular}[0.7]{p{7.25cm} >{\centering\arraybackslash}p{4cm} >{\centering\arraybackslash}p{4cm}}
				Нормоконтроллер & \uline{\hspace*{4cm}} & \uline{\hfill \textbf{<И.О. Фамилия>} \hfill} \\
				& \scriptsize (Подпись, дата) & \scriptsize (И.О. Фамилия)
			\end{signstabular}
		\end{table}

		\vfill

		\begin{center}
			\normalsize \textit{\textbf{<Год>} г.}
		\end{center}
	\end{titlepage}

	\normalsize
	\setcounter{page}{4}
	\section*{АННОТАЦИЯ}

	\pagebreak

	\section*{РЕФЕРАТ}
		\begin{center}
			Расчетно-пояснительная записка \pageref{LastPage} с., \totalfigures\ рис., \totaltables\ табл., Х ист., Х прил.

			\textbf{<Ключевые слова>}
		\end{center}

		\pagebreak

	% Переопределяем название \toc и выводим сам \toc
	\renewcommand{\contentsname}{\normalsize\bfseries\centering СОДЕРЖАНИЕ}
	\small
	\tableofcontents
	\normalsize

		\pagebreak

	\section*{ОПРЕДЕЛЕНИЯ, ОБОЗНАЧЕНИЯ И СОКРАЩЕНИЯ}

		\pagebreak

	\section*{ВВЕДЕНИЕ}
		\addcontentsline{toc}{section}{ВВЕДЕНИЕ}

		\pagebreak

	\section{<Название исследовательской части>}

		\pagebreak

	\section{<Название конструкторской части>}

		\pagebreak

	\section{<Название технологической части>}

		\pagebreak

	\section*{ЗАКЛЮЧЕНИЕ}
		\addcontentsline{toc}{section}{ЗАКЛЮЧЕНИЕ}

		\pagebreak

	\section*{СПИСОК ИСПОЛЬЗОВАННЫХ ИСТОЧНИКОВ}
		\addcontentsline{toc}{section}{СПИСОК ИСПОЛЬЗОВАННЫХ ИСТОЧНИКОВ}

		\pagebreak
\end{document}
